\documentclass[UTF8]{ctexart}

\usepackage{myTemp}
\lhead{他励直流电动机双闭环调速系统设计}

\begin{document}
\pagenumbering{arabic}
%%%%%%%%%%%%%%%%%%%%%%%%%%%%封面与目录%%%%%%%%%%%%%%%%%%%%%%%%%%%%%%
\begin{titlepage}
  \begin{center}
  % Upper part of the page
  \includegraphics[width=0.25\textwidth]{imgs/logo.jpg}\\[1cm]
  \textsc{\LARGE Department of Automation}\\[1.5cm]
  \fangsong{\Large 电力拖动与运动控制仿真实验}\\[0.5cm]
  % Title
  \hrulefill
  \\[0.8cm]{\centering \LARGE \heiti 他励直流电动机双闭环调速系统设计}\\[0.4cm]
  \hrulefill
  \\[4cm]

  % Author and supervisor
  \begin{tabular}{ccc}
  \fangsong\sihao\textbf {班\hspace{1.5cm}级}&\ :\ &\fangsong\sihao\textbf{自~~3~2}\\
  \\
  \fangsong\sihao\textbf {姓\hspace{1.5cm}名}&\ :\ &\fangsong\sihao\textbf{陈~~~昊~~~楠}\\
  \\
  \fangsong\sihao\textbf {学\hspace{1.5cm}号}&\ :\ &\fangsong\sihao\textbf{2013011449}\\
  \\
  \end{tabular}
  \vfill
  {\large \today}
  \end{center}
\end{titlepage}
\clearpage

\thispagestyle{plain}
\begin{center}
    \heiti\xiaosanhao{中文摘要}
    
    \vspace{0.4cm}
    \large
    Thesis Subtitle
    
    \vspace{0.4cm}
    \textbf{Author Name}
    
    \vspace{0.9cm}
    \textbf{Abstract}
\end{center}

\tableofcontent

%%%%%%%%%%%%%%%%%%%%%%%%%%正文部分%%%%%%%%%%%%%%%%%%%%%%%%%%%%%%%%%%

该问题是一个积分型性能指标,末态有约束的最优控制问题。初始条件 $p(0)=0$,$v(0)=v^0$(给定值),末态约束 $p(t^\mathrm{m})=L$,$v(t^\mathrm{m})=v^\mathrm{m}$(给定值),状态方程 $\dot{p}=v$,$\dot{v}=u$,控制函数的约束条件为 $|u|\leq u_\max$,求最优控制函数,使性能指标

\begin{equation}
J=\int_{t^0}^{t^\mathrm{m}} u^2(t)\mathrm{d}t
\end{equation}
达到极小。

该问题的哈密顿函数为
\begin{equation}
H=\frac12u^2+\lambda^\mathrm{p}v+\lambda^\mathrm{v}u_i
\end{equation}

% \section{实验目的}
% \begin{bracketiter}
% \item  熟悉仿真软件Matlab/Simulink的基本使用方法及其在电力电子与电力拖动控制系统中的应用;
% \item  掌握直流电动机双闭环调速系统中调节器的工程设计方法;
% \item  讨论负载特性对本实验中系统控制性能的影响,对比系统对不同转速指令信号的跟踪特性。
% \end{bracketiter}

% \section{实验内容}
% \begin{bracketiter}
% \item 电流调节器与转速调节器的设计;
% \item 基于Matlab/Simulink搭建直流电动机双闭环调速系统的仿真模型;
% \item 仿真观测直流电动机双闭环调速系统正常工况下的运行特性;
% \item 对比不同负载情况对双闭环调速系统控制性能的影响;
% \item 对比双闭环调速系统对不同转速指令跟踪性能的差别,分析可能的原因,并对系统进行改进。
% \end{bracketiter}

% \section{实验任务}


  \item 确定时间常数。
  \begin{enumerate}[label=(\alph*)]
  \item 电流环等效时间常数。由于电流环按典型\romannum{1}型系统设计,且参数选择为$K_\mathrm{op,i}T_\mathrm{\Sigma i}=0.5$,因此电流环等效时间常数为$2T_\mathrm{\Sigma i}=\num{2x 0.00367}=\SI{0.00734}{s}$。
  \item 转速环小时间常数$T_\mathrm{\Sigma n}$。已知转速滤波时间常数为$T_\mathrm{on}=\SI{0.01}{s}$,因此转速环小时间常数为,
  \begin{equation}
  T_\mathrm{\Sigma n}=2T_\mathrm{\Sigma i} + T_\mathrm{on}=\num{0.00734}+0.01 = \SI{0.01734}{s}
  \end{equation}
  \end{enumerate}
  \item 确定转速调节器结构和参数。
  \begin{enumerate}[label=(\alph*)]
  \item 结构选择,由于设计要求无静差,因此转速调节器必须含有积分环节。又考虑到动态要求,转速调节器应该采用PI调节器,按典型\Romannum{2}型系统设计。转速调节器的传递函数为,
  \begin{equation}
  W_\mathrm{ASR}(s) = \frac{K_\mathrm{n}(\tau_\mathrm{n}s+1)}{\tau_\mathrm{n}s}
  \end{equation}
  \item 参数计算,依据给定的可变参数,取中频宽$h=9$,按照$\gamma_{\max}$准则确定参数关系,ASR的超前时间常数为,
  \begin{equation}
  \tau_\mathrm{n}=hT_\mathrm{\Sigma n}=\num{9x0.01734}=\SI{0.15606}{s},
  \end{equation}
  转速环开环放大系数为,
  \begin{equation}
  K_\mathrm{op,n} = \frac1{h\sqrt{h}T^2_\mathrm{\Sigma n}}=\frac1{9\times\sqrt{9}\times\num{0.01734}^2}=\SI{123.1794}{s^{-2}}
  \end{equation}
  转速调节器的比例放大系数为
  \begin{equation}
  K_\mathrm{n}=\frac{\beta C_\mathrm{e}T_\mathrm{m}}{\sqrt{h}\alpha RT_{\Sigma n}}=\frac{\num{0.05x0.132x0.18}}{\sqrt{9}\times\num{0.007 x 0.5 x 0.01734}}=6.525
  \end{equation}
  \end{enumerate}
  \item 校验近似条件和性能指标。
  \begin{enumerate}
  \item 电流环传递函数等效条件$\omega_\mathrm{cn}\le \frac1{5T_\mathrm{\Sigma i}}$。由$K=\omega_1\omega_\mathrm{c}$,按$\gamma_{\max}$ 准则,可以求得转速环截止频率$\omega_\mathrm{cn}$为
  \begin{equation}
  \omega_\mathrm{cn} = \frac{K_\mathrm{op,n}}{\omega_1} = K_\mathrm{op,n}\tau_{n} = \num{123.2x0.15606} = \SI{19.227}{s^{-1}}
  \end{equation}
  而,$1/5T_\mathrm{\Sigma i} = \SI{54.5}{s^{-1}}>\omega_{cn}$,满足等效条件。
  \item 转速环小时间常数近似处理条件$\omega_\mathrm{cn}\le \frac13\sqrt{\frac1{2T_\mathrm{\Sigma i}T_\mathrm{on}}}=\SI{38.9}{s^{-1}}$,满足近似处理条件。
  \end{enumerate}
  \end{enumerate}
  \end{bracketiter}

  \paragraph{模型初始化}
  在{\ttfamily Simulink}中导入模型,按照计算得到的调节器参数对模型进行初始化。模型如图\ref{fig:model}。电流调节器ACR与转速控制器分别如图\ref{fig:acr}与\ref{fig:asr}。

  \begin{figure}[htbp]
  \centering
  \includegraphics[width=15cm]{imgs/model.png}
  \caption{仿真模型图}
  \label{fig:model}
  \end{figure}

  \begin{figure}[htbp]
  \centering
  \includegraphics[width=10cm]{imgs/acr.png}
  \caption{电流调节器仿真模型,传递函数$W_\mathrm{ACR}=\frac{1.0220(0.03s+1)}{0.03s}$}
  \label{fig:acr}
  \end{figure}

  \begin{figure}[htbp]
  \centering
  \includegraphics[width=10cm]{imgs/asr.png}
  \caption{转速调节器仿真模型,传递函数$W_\mathrm{ASR}=\frac{6.525(0.156s+1)}{0.156s}$}
  \label{fig:asr}
  \end{figure}

  工作区其他参数设置如图\ref{fig:params}。

  \begin{figure}[htbp]
  \centering
  \includegraphics[width=5cm]{imgs/params.png}
  \caption{模型初始参数设置}
  \label{fig:params}
  \end{figure}

\end{document}
